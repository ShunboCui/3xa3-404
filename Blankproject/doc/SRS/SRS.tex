\documentclass[12pt, titlepage]{article}

\usepackage{amssymb}
\usepackage{amstext}
\usepackage{amsthm}
\usepackage{amsmath}
\usepackage{changepage}
\usepackage{enumerate}
\usepackage{fancyhdr}
\usepackage[margin=1in]{geometry}
\usepackage{graphicx}
\usepackage{extarrows}
\usepackage{setspace}
\usepackage{tikz}
\usepackage{color}
\usepackage{booktabs}
\usepackage{tabularx}
\usepackage{hyperref}
\usepackage{float}
\hypersetup{
    colorlinks,
    citecolor=black,
    filecolor=black,
    linkcolor=red,
    urlcolor=blue
}
\usepackage[round]{natbib}

\title{SE 3XA3: Software Requirements Specification\\\textbf{Snake Game Remake}}

\author{Team \#: 302, Team Name: 404
		\\ Student: Shunbo Cui	cuis13
		\\ Student: Xiangxin Kong	kongx9
		\\ Student: Shuo Zhang	zhans18
}


\date{February 9, 2020}

%\input{../Comments}

\begin{document}

\maketitle
\pagenumbering{roman}
\tableofcontents
\listoftables
\listoffigures

\begin{table}[bp]
\caption{\bf Revision History}
\begin{tabularx}{\textwidth}{p{3cm}p{2cm}X}
\toprule {\bf Date} & {\bf Version} & {\bf Notes}\\
\midrule
Feb 9,2020 & 1.0 & creating the document\\
Date 2 & 1.1 & Notes\\
\bottomrule
\end{tabularx}
\end{table}

\newpage

\pagenumbering{arabic}

This document describes the requirements for ....  The template for the Software
Requirements Specification (SRS) is a subset of the Volere
template~\citep{RobertsonAndRobertson2012}.  If you make further modifications
to the template, you should explicity state what modifications were made.

\section{Project Drivers}

\subsection{The Purpose of the Project}
The purpose of the software application, Snake-game-remake, is to provide an enjoyable and challenging gaming experience for the user. It is inspired by the games \href{https://github.com/mtala3t/Snake-Java-2D-Game}{Snake-Java-2D-game} in that it focuses on the growth of a snake character through consuming objects throughout the in-game environment. The snake will be a dynamic entity that moves automatically with a constant speed, unless it gains the effect from specific items. The player can adjust the direction to left or right. The player has to operate the snake to avoid collision with the walls and barriers in the map, while also collect the food items in the map to achieve higher score. In order to improve the complexity of the game, more in-game functions will be provided such as random potions and trap items.
\subsection{The Stakeholders}

\subsubsection{The Client}

\subsubsection{The Customers}

\subsubsection{Other Stakeholders}

\subsection{Mandated Constraints}

\subsection{Naming Conventions and Terminology}

\subsection{Relevant Facts and Assumptions}

User characteristics should go under assumptions.

\section{Functional Requirements}

\subsection{The Scope of the Work and the Product}

\subsubsection{The Context of the Work}
Some important functions are absent in this game such as leader board and restart option when game is over. Also, the choices of maps for this game are limited and the original design for the interface is rough and simple. We are going to reconstruct the user interface to enhance user experience. More gaming systems such as utility items will be introduced to the game to improve the complexity of the game. Our game is a light weighted Java game on PC which can be played by all ages especially teenagers. All the additional functions will be implemented in individual Java modules with high cohesion.
\subsubsection{Work Partitioning}
	\begin{table}[H]
	\caption{Work Partitioning}
        \begin{adjustwidth}{-2.75cm}{}
    \begin{tabular}{|c|c|c|p{5cm}|}
    \hline
    Event Name & Input & Output & Summary\\
    \hline
    Snake Game Creation & Developer code & Java Virtual Machine & Recreate a Java based game implementation that works on Java Virtual Machine.\\
    \hline
    Snake Game Audio & Audio File & Audio output device & Adding more sound effects into the game.\\
    \hline
    Snake Game Sprite & Developer graphics and code & Java Virtual Machine & Adding more functions of the spirits and objects to the game.\\
    \hline
    Snake Game Collisions & Developer code & Java Virtual Machine & Creating hit detection and ending screen. \\
    \hline
    Snake Game Final Edits & Developer code & Java Virtual Machine & Finishing edits for the whole project.\\
    \hline
    \end{tabular}
	    \end{adjustwidth}
	\end{table}
\subsubsection{Individual Product Use Cases}
The product is designed to be used by those teenagers who enjoy video game experience. As a successful game on Nokia phone in 1990s, it requires t both quick reactions and forethough. Our product will provide improved experience of challenging themselves to get higher score on the leader board. As an open-source project on public Git repository, our code also provides educational support to the developers who are interested in similar projects.
\subsection{Functional Requirements}
\begin{enumerate}[{FR}1.]
    \item The program will be able to run on Java virtual machine.
    \item The game will show the main title screen when the program is successfully started.
    \item The system will adjust the speed of the snake and the maximum number of in-game items when the user select different difficulty on the title screen.
    \item The player will be able to choose the game map they want to play on the title screen.
    \item When the start game button on the title screen is pressed, a new game will be started and the gaming screen will be shown.
    \item At the start state of the game, a snake with length of 2 and default direction to the right will be generated in the middle of the map.
    \item The snake will start moving when the player presses the space button and a sound effect will be played.
    \item A food item will be generated at random empty location inside the border of map after the player push space button to start.
    \item When the player presses pause button, the pause screen will be shown.
    \item When the player presses pause button on the pause state, the playing screen will be shown again.
    \item On the pause screen, a list of buttons including go back to title screen and restart game will be shown.
    \item Every time when the snake's head reaches the existing food item, the length of the snake will be incremented by 1 at the tail and the score will be incremented, also the sound effect will be played.
    \item The food collected by the snake will be removed from the map and a new food item will be generated at random empty location at the same time.
    \item The current length, effect and score should be always shown on the gaming screen.
    \item After a set time period, a new effect item will be generated on a random empty location on the map until the maximum number of items is reached.
    \item When the snake reaches the effect item, the item is removed from the map and the corresponding effect will be applied on the snake; if there is already effect on the snake, it will be replaced with the new one.
    \item When the snake's head hit the border wall, the barrier in the map or the snake body, the game will be stopped and a game-over screen will be shown.
    \item The player will be able to restart the game or go back to title screen with the buttons on the game-over screen.
    \item The player will be able to activate color blind option with a button on the title screen.
    \item The colors used in the game will be adjusted to be better visible by  colourblind people when the color-blind mode is activated.
\end{enumerate}
\section{Non-functional Requirements}

\subsection{Look and Feel Requirements}
\begin{enumerate}[{LF}1. ]
	\item The game will follows Java coding structure.
	\item The game shall be two-dimensional in design.
	\item The game should use colors that make players  easily identify the objects in the game.
\end{enumerate}
\subsection{Usability and Humanity Requirements}
%%%%%%%%%%%%%%%%%%%%%%%%%%%%%%%%%%%%%%%%%
\begin{enumerate}[{UH}1. ]
	\item  Players shall be able to customize the color for the the snake before playing.
	\item There shall be a help page available for users to find the controls and descriptions of items.
	\item The game shall provide the objective of the game in the homepage.
	\item The game shall be accessible for deaf users.
	\item The game shall be playable by users with at least a single fully functional hand.
\end{enumerate}
%%%%%%%%%%%%%%%%%%%%%%%%%%%%%%%%%%%%%%%%%
\subsection{Performance Requirements}

\subsection{Operational and Environmental Requirements}

\subsection{Maintainability and Support Requirements}

\subsection{Security Requirements}
\begin{enumerate}[{SR}1. ]
	\item The game shall not pass any variable, command or program to the player’s local machine.
	\item The game shall not store any information of the player’s local machine.
\end{enumerate}

\subsection{Cultural Requirements}
\begin{enumerate}[{CR}1. ]
	\item The system shall not show any religious imagery or messages.
\end{enumerate}
\subsection{Legal Requirements}
\begin{enumerate}[{LR}1. ]
	    \item This software shall comply with all national and federal software regulation laws.
        \item This software shall comply with all relevant software standards.
        \item This software shall comply with all relevant privacy acts.
        \item Design shall follow Google Java Style Guide
\end{enumerate}
\subsection{Health and Safety Requirements}
\begin{enumerate}[{HS}1. ]
	    \item Cacophony shall not be selected as the game sound.
	    \item There shall be tips in the game that remind the player not to be addicted to playing games
\end{enumerate}


\section{Project Issues}

\subsection{Open Issues}
    The renewal period of software and hardware shorten continuously. These changes may make the game no longer playable or run with bugs. Apart from that, many devices do not have physical keyboard. Users of these devices may not be able to play this game.
\subsection{Off-the-Shelf Solutions}
    Snake game is a very classic game that has been implemented in many languages. These products have the same core game rules game but different add-ons. Some of them add multiplayer options in the game. Some others focus on adding more scenarios, game modes or more items. A java based product called "Snake2D" can be used as the prototype of this project. Other products on GitLab can be treated as source of code.
\subsection{New Problems}
%%%%%%%%%%%%%
\subsubsection{current environment}
    Although most of the snake game online are open source project, there are still many snake game with copyrights. The developing team should be careful not to breach copyright laws.
\subsubsection{existing users}
    Color-blind users may have a hard time identifying the snake and items in the game.
\subsubsection{Limitations in implementation environment}
    The java virtual machine on the local machine of the developing team processes the code slowly. (long processing time).
\subsubsection{problem from "solution"}
    None.
%%%%%%%%%%%%%
\subsection{Tasks}
    The plan for the delivery of the project and each member's role in every delivery are covered by the Gantt chart located in the "project schedule" file in the project. The plan for the meeting is covered in "development plan". In the Gantt chart, phase one tasks are in light blue color that shows the developing team will be working on structuring the project. Phase two tasks are in red color that shows the team will be constructing and building tests for the project. Also, there will be changes in requirements in phase two. The final phase tasks are colored in yellow that shows the developing team has completed the coding part of the project and will be in progress in modifying documents and code for the final product.

\subsection{Migration to the New Product}
    None.\\This project will only be used in the device that supports java.
\subsection{Risks}
    None. As a software product, there is no physical damage to users. However there might be a risk that some users will be addicted to the game.
\subsection{Costs}
    As this project will be available on GitLab, there will be no cost for downloading the game. But in order to play this game, users need a device that support java and be able to connect a keyboard.
\subsection{User Documentation and Training}
    Instructions will be shown in the game. There is no need for a separate user documentation file to guide users. Users can get training by playing this game.
\subsection{Waiting Room}
	A possible improvement of the original project multiplayer option can be added to the game to let two players compete for food and items. This will make the game more competitive and attractive to the target users of this product. Then, the game only has one music sound that is played throughout the game. Adding more sounds will also improve the user experience.

\subsection{Ideas for Solutions}
    For the multiplayer option, the second snake feature can be introduced into the game interface with different control buttons and color from the first snake. The two snakes will compete on their scores gaining from eating food. They will also compete for items and avoid hitting on each other. For new sounds, they will be selected from online music records that have no copyright. Background music shall differ in different maps.
\bibliographystyle{plainnat}

\bibliography{SRS}

\newpage

\section{Appendix}

This section has been added to the Volere template.  This is where you can place
additional information.

\subsection{Symbolic Parameters}

The definition of the requirements will likely call for SYMBOLIC\_CONSTANTS.
Their values are defined in this section for easy maintenance.


\end{document}
